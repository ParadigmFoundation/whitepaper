\documentclass[10pt]{article}
\usepackage{graphicx}
\usepackage{hyperref}
\usepackage{amsmath}
\usepackage{enumitem}
\usepackage{cite}
\usepackage[labelfont=bf]{caption}
\usepackage{float}
\usepackage{listings}
\usepackage{lstautogobble}
\usepackage{adjustbox}
\usepackage{tabularx}

\lstset{basicstyle=\ttfamily,
  mathescape=true,
  escapeinside=||,
  autogobble}

\input{solidity-highlighting.tex}

% wide margins
\addtolength{\oddsidemargin}{-.75in}
\addtolength{\evensidemargin}{-.75in}
\addtolength{\textwidth}{1.5in}
\addtolength{\topmargin}{-.875in}
\addtolength{\textheight}{1.75in}

\captionsetup[table]{skip=4pt}

\begin{document}

% START TITLE PAGE
\title{ \textbf{Kosu: A Decentralized Relay Protocol for Smart Contract Based Financial Primitives} } 
\author{Liam Kovatch, Henry Harder \\\texttt{https://kosu.io}}
\date{5 June 2018 (Revised \today)}
\maketitle 
% END TITLE PAGE

%%%%%%%%%%%%%%%%%%%%%%%%%%%%%%%%%%%%%%%%%%%%%%%%%%%%%%%%%%%%%%%%%%%%%%%%%%%%%%%%
%~~~~~~~~~~~~~~~~~~~~~~~~~~~~~~~~~~~~~~~~~~~~~~~~~~~~~~~~~~~~~~~~~~~~~~~~~~~~~~%
%%%%%%%%%%%%%%%%%%%%%%%%%%%%%%%%%%%%%%%%%%%%%%%%%%%%%%%%%%%%%%%%%%%%%%%%%%%%%%%%v

%BEGIN ABSTRACT
\begin{abstract}

\noindent We motivate Kosu, a decentralized network and protocol that encourages the curation of a consistent, high-quality, verifiable, public order book. The Kosu network is an independent bonded proof-of-stake blockchain built on Tendermint Consensus. The lightweight state application is responsible for enforcing a simple access control rule-set and order booking procedure. The network utilizes a shared security model, wherein all staking and token-based mechanics happen on the Ethereum main-network. The model provides unidirectional communication and checkpoints between the two networks by leveraging Tendermint to provide finality for specific Ethereum-based state changes. Individuals intending to post orders to the Kosu network are required to bond tokens for the duration they wish to have access. The poster bonding mechanism creates a market for the allocation of network resources throughput and an implicit incentive to contribute quality liquidity to the network. Kosu validators are selected via a token curated registry system on Ethereum. This registry system allows both the inflation rate as well as the number of active validators to be market driven and determined by the network’s stakeholders. The network is highly generalizable, supporting orders of arbitrary maker-taker based virtual settlement systems. In its entirety, the proposed system effectively curates a decentralized order book that serves as a liquidity aggregation primitive for second layer systems. Kosu will be free to use and open source, reducing barriers to entry for both developers and market agents. // TODO
\medskip

\end{abstract}
\pagebreak
% END ABSTRACT

%%%%%%%%%%%%%%%%%%%%%%%%%%%%%%%%%%%%%%%%%%%%%%%%%%%%%%%%%%%%%%%%%%%%%%%%%%%%%%%%
%~~~~~~~~~~~~~~~~~~~~~~~~~~~~~~~~~~~~~~~~~~~~~~~~~~~~~~~~~~~~~~~~~~~~~~~~~~~~~~%
%%%%%%%%%%%%%%%%%%%%%%%%%%%%%%%%%%%%%%%%%%%%%%%%%%%%%%%%%%%%%%%%%%%%%%%%%%%%%%%%

% TABLE OF CONTENTS
\tableofcontents
\pagebreak
% END TABLE OF CONTENTS

%%%%%%%%%%%%%%%%%%%%%%%%%%%%%%%%%%%%%%%%%%%%%%%%%%%%%%%%%%%%%%%%%%%%%%%%%%%%%%%%
%~~~~~~~~~~~~~~~~~~~~~~~~~~~~~~~~~~~~~~~~~~~~~~~~~~~~~~~~~~~~~~~~~~~~~~~~~~~~~~%
%%%%%%%%%%%%%%%%%%%%%%%%%%%%%%%%%%%%%%%%%%%%%%%%%%%%%%%%%%%%%%%%%%%%%%%%%%%%%%%%

% BEGIN INTRO
\section{Introduction}\label{intro}

\noindent Historically, financial markets have been relatively siloed. To date, centralized entities control significant entry points and accounting systems, limiting access beyond a small set of privileged participants. The recent invention of blockchains, decentralized ledger systems, is now challenging this trusted paradigm. Ethereum, a public blockchain with a built-in Turing-complete programming language, provides developers an immutable and verifiable computation platform for scripts known as “smart-contracts”. These scripts can encode arbitrary, deterministic state transition functions, providing a powerful primitive for more complex ledger based application systems.
\medskip

\noindent Since the network’s genesis in 2014, developers have leveraged smart contracts for a wide range of novel protocol systems. Perhaps the most adopted technical standard for smart contracts is the ERC-20 token implementation. The standard defines a common token interface allowing for the creation of a diverse set of Ethereum-based assets. These tokens often serve as a critical crypto economic primitive providing core utility for protocol systems also deployed on Ethereum. As such, a system of exchange is a key requisite in a mature token-dependent contract ecosystem.
\medskip

\noindent In early 2017, the 0x Project published a whitepaper defining a maker-taker based exchange  protocol in which a pipeline of publicly accessible Ethereum smart contracts facilitate the wallet-to-wallet settlement of funds. In this model, order broadcast and discovery happen off-chain. Independent entities known as relayers are responsible for sourcing and maintaining liquidity in the form of independent collections of signed order messages. This hybrid architecture drastically improved the efficiency of existing DEX systems while also providing a critical open exchange primitive with high extensibility. Since the first deployment of the 0x protocol, a significant number have adopted it as a common piece of infrastructure. To date, the team continues to push the limits on scalability, efficiency and incentivization for DEX systems.
\medskip

\noindent
Virtual settlement systems, specifically decentralized exchanges (DEXs), are a relatively accessible use case for blockchains; a direct evolution from the original digital currency use-case. While conceptually apparent, the architecture of DEXs is a complex and evolving subject. The scalability restraints of blockchain systems combined with the complexities of an asynchronous discrete time system limit their compatibility with optimally efficient continuous time order books and settlement systems. As such, alternative DEX architectures have also captured considerable market share. These systems include dutch auctions and automated market making systems. While immediately compelling for some use cases, the evolution towards hybrid continuous time order books is likely inevitable. These systems, though currently constrained, likely represent the most scalable and efficient DEX structure.
\medskip

\noindent
The Kosu network is designed within the 0x-motivated ecosystem, providing a decentralized aggregation primitive for arbitrary limit orders. More specifically, the Kosu network aims to provide a highly extensible decentralized complement to independent centralized aggregators, or relayers. The set of orders maintained on the Kosu network should represent a high-value subset of the compatible order universe.

\clearpage
\pagebreak
% END INTRO

%%%%%%%%%%%%%%%%%%%%%%%%%%%%%%%%%%%%%%%%%%%%%%%%%%%%%%%%%%%%%%%%%%%%%%%%%%%%%%%%
%~~~~~~~~~~~~~~~~~~~~~~~~~~~~~~~~~~~~~~~~~~~~~~~~~~~~~~~~~~~~~~~~~~~~~~~~~~~~~~%
%%%%%%%%%%%%%%%%%%%%%%%%%%%%%%%%%%%%%%%%%%%%%%%%%%%%%%%%%%%%%%%%%%%%%%%%%%%%%%%%

% BEGIN OVERVIEW SECTION
\section{Overview}\label{overview}

\noindent Kosu is chiefly composed of two interdependent systems that together implement the protocol’s core functionality.

\medskip

\noindent The first system is the Kosu blockchain, a collection of globally distributed nodes that collectively maintain and update the network’s shared order book\footnote{The Kosu order book is not a conventional limit order book, but rather a hash of specific recent orders from posters with active bandwidth allocation bonded tokens. The structure described in the ORDERBOOK // TODO section allows for a proof to be generated about the validity of a set of individual orders at a given Kosu block height.  See more here // TODO: link}, and maintain a one-way Peg with the Ethereum blockchain. The network supports basic order booking functionality, access control for user accounts, and cryptographic proofs of the existence of state contents, including the set of orders comprising the "order book". Validators\footnote{ The networks full nodes provide access to the order book and provide an interface for users to submit orders, laying the foundation for a wide array of applications.} secure the network by voting on state changes to an Ethereum contract system, processing orders, and reaching consensus with other validators on the canonical chain and the overall application state.

\medskip

\noindent The second and closely related component is the Kosu protocol smart-contract system. These contracts are deployed to Ethereum and implement Kosu’s native token and incentive mechanism, support access control and sybil tolerance for posters, and allow stakeholders to curate the Kosu network’s validator set.

\medskip

\noindent Kosu validators run full Ethereum nodes and are responsible for submitting special attestation transactions to specific state changes to the Kosu contract system. The Kosu networks knowledge about the state of the contract system is limited to only what is strictly necessary, such as maintaining a registry of users allowed to post orders to the network (posters).

\medskip

\noindent Posters who wish to leverage the network’s decentralized order booking features can gain write access to the network by bonding any amount of Kosu tokens in a specific contract. After their bond transaction is confirmed on Ethereum, validators update the networks shared state to reflect the bonded balance change of the new poster. The same mechanism allows posters to adjust the amount of tokens they have bonded, or withdraw entirely from the system at any time.

\medskip

\noindent At deterministic intervals based on the height of the Ethereum blockchain, Kosu validators compute a simple account-limit mapping based on in-state data that allocates network throughput proportionally to posters who have Kosu tokens bonded at the height a new interval starts. These intervals are called rebalance periods and allow the continuous process of balance updates on the Ethereum Poster Registry as a result of bonding/unbonding by posters to be mapped to discrete periods of bandwidth limiting in a deterministic manner. At the beginning of each new rebalance period, validators must reach consensus on the parameterization of the upcoming period. The starting and ending Ethereum block heights of the period, and the total number of orders to be accepted from posters during that time is determined prior to allowing posters write access for the period.

\medskip

\noindent During normal network operations between rebalance periods, validators accept and process incoming order transactions from posters that have been gossiped to them through full nodes, while concurrently submitting regular attestations (called witness transactions) to other validators on the network to ensure the state of the Kosu Ethereum contract system is accurately tracked.

\medskip

\noindent Order transactions are verified purely based on the signature of the poster that submitted the order to the network, and some basic data structure 
requirements that enable signature recovery. If a valid signature is included in the incoming Order and the recovered address matches that of a poster account with a non-zero remaining limit for the current rebalance period, validators will accept the order and include it in a block and decrement the posters bandwidth allocation limit for that period. Otherwise, the order is rejected.

\medskip

\noindent ... DESCRIBE ORDER BOOK AND QUERY MODEL, PROOFS ETC
\clearpage
\pagebreak
% END OVERVIEW SECTION

%%%%%%%%%%%%%%%%%%%%%%%%%%%%%%%%%%%%%%%%%%%%%%%%%%%%%%%%%%%%%%%%%%%%%%%%%%%%%%%%
%~~~~~~~~~~~~~~~~~~~~~~~~~~~~~~~~~~~~~~~~~~~~~~~~~~~~~~~~~~~~~~~~~~~~~~~~~~~~~~%
%%%%%%%%%%%%%%%%%%%%%%%%%%%%%%%%%%%%%%%%%%%%%%%%%%%%%%%%%%%%%%%%%%%%%%%%%%%%%%%%

% BEGIN TECH SPEC SECTION
\section{Specification}\label{specification}

%%%%%%%%%%%%%%%%%%%%%%%%%%%%%%%%%%%%%%%%%%%%%%%%%%%%%%%%%%%%%%%%%%%%%%%%%%%%%%%%
% BEGIN ORDER SPECIFICATION SUB-SECTION
\subsection{Orders}\label{orders}

\noindent The Kosu protocol defines a simple and extensible data structure to represent signed order messages on the network. The primary purpose of the defined order schematic is to allow signature recovery for the verification of poster bandwidth allocations, during the network’s order verification process.
\medskip

\noindent The order format described below is also designed to allow the “wrapping” of already existing hybrid-decentralized order message formats (such as 0x and Dharma) for relay on the Kosu network, and settlement through a system of generalizable forwarding contracts. 
\medskip 

\noindent Usage of the forwarding contract system is strictly optional, and unrelated to the core protocol in the sense that no state is shared between the protocol contract system, and settlement contracts.

\subsubsection{SubContract interface}\label{subcontract-interface}

\noindent Kosu can act as an order message aggregator and/or transport layer for a variety of types of on-chain exchange systems. To achieve this, a simple and extensible contract-based interface is defined that allows a common order message format to be used for a variety of Ethereum settlement pipelines that leverage hybrid decentralized exchange architecture. // link to definition of this term
\medskip

\noindent All current SubContract implementations are in Solidity and designed for use with Ethereum, however, any language and blockchain that supports contract-like settlement implementations that satisfy the interface above, and the correct signature scheme, can be used with Kosu.
\medskip

% Begin table 1
\bgroup
\def\arraystretch{1.2}
\begin{table}[H]
\centering
\label{table:table1}
\begin{tabularx}{1\textwidth}{|l|l|X|p{8cm}|}
\hline
\textbf{Method name} & \textbf{Returns} & \textbf{Params.} & \textbf{Description} \\ 
\hline
\texttt{makerArguments} & \texttt{string} & \texttt{-} & Returns a JSON string containing all the fields required for maker orders. \\
\hline
\texttt{takerArguments} & \texttt{string} & \texttt{-} & Returns a JSON string containing all the fields a taker must supply when filling a maker order. \\
\hline
\texttt{isValid} & \texttt{boolean} & \texttt{makerData} & Checks weather a maker order is valid and fillable based on the SubContract's validation implementation. \\
\hline
\texttt{amountRemaining} & \texttt{uint256} & \texttt{makerData} & For settlement types that support it, this method can return information that allows partial fills \\
\hline
\texttt{participate} & \texttt{boolean} & \texttt{makerData}, \texttt{takerData} & The main settlement logic implementation for SubContracts, which triggers execution of a trade by the taker submitting the maker data and their counter-party information. \\
\hline
\end{tabularx}
\caption{Describes the SubContract interface and method signatures in a language-independent manner.} 
\end{table}
\egroup
% End table 1

\subsubsection{Order message format}\label{order-message-format}

\noindent A simple order format is defined based on the SubContract interface, and the requirements for signature verification... // TODO
\medskip

% Begin table 2
\bgroup
\def\arraystretch{1.2}
\begin{table}[H]
\centering
\label{table:table2}
\begin{tabularx}{1\textwidth}{|l|l|l|X|}
\hline
\textbf{Field name} & \textbf{Type} & \textbf{Required} & \textbf{Description} \\
\hline
\texttt{subContract} & \texttt{string} & yes & The deployed address of the target SubContract settlement implementation. Defines expected arguments. \\
\hline
\texttt{maker} & \texttt{string} & yes & The address of the party that signed the maker order. Usually indicates the beneficiary of settled funds. \\
\hline
\texttt{makerArguments} & \texttt{array} & no & An array of equal length to the number of \texttt{makerArguments} containing objects that define the name and data-type for each argument. \\
\hline
\texttt{takerArguments} & \texttt{array} & no & Similar to \texttt{makerArguments}, it defines the values and data-types required for settlement. Not required for maker orders. \\
\hline
\texttt{makerValues} & \texttt{object} & yes & A hash-map data structure that contains the parameters necessary for a valid maker order of the target settlement type. \\
\hline
\texttt{makerSignature} & \texttt{object} & no & An optional field that can be used to include a signature from the maker. May also be included in \texttt{makerValues}. \\
\hline
\texttt{posterSignature} & \texttt{object} & yes & Stores the signature resulting from a poster entity signing a hash of the maker order values. Used to verify poster has bonded tokens. \\
\hline
\end{tabularx}
\caption{Generically describes the data structure that represents an order message.} 
\end{table}
\egroup
% End table 2
% END ORDERS SUB-SECTION
%%%%%%%%%%%%%%%%%%%%%%%%%%%%%%%%%%%%%%%%%%%%%%%%%%%%%%%%%%%%%%%%%%%%%%%%%%%%%%%%

%%%%%%%%%%%%%%%%%%%%%%%%%%%%%%%%%%%%%%%%%%%%%%%%%%%%%%%%%%%%%%%%%%%%%%%%%%%%%%%%
% BEGIN ETHEREUM CONTRACT SYSTEM SUB-SECTION
\subsection{Ethereum Contract System}\label{ethereum-contract-system}
\subsubsection{Introduction}\label{ethereum-contract-system-intro}
\noindent The Kosu Ethereum contract system...
\bigskip

\subsubsection{Permission model}\label{ethereum-contract-system-auth}
\noindent Access control within...
\bigskip

\subsubsection{Architecture}\label{ethereum-contract-system-architecture}
\noindent The contract system uses a modular... 
\bigskip

\subsubsection{Treasury}\label{ethereum-contract-system-treasury}
\noindent The Kosu treasury contract
\bigskip

\subsubsection{Poster bonding}\label{ethereum-contract-system-poster-bonding}
\noindent Posters can...
\bigskip

\subsubsection{Validator token-curated registry}\label{ethereum-contract-system-tcr}
\noindent The validator set can...
% END ETHEREUM CONTRACT SYSTEM SUB-SECTION
%%%%%%%%%%%%%%%%%%%%%%%%%%%%%%%%%%%%%%%%%%%%%%%%%%%%%%%%%%%%%%%%%%%%%%%%%%%%%%%%

%%%%%%%%%%%%%%%%%%%%%%%%%%%%%%%%%%%%%%%%%%%%%%%%%%%%%%%%%%%%%%%%%%%%%%%%%%%%%%%%
% BEGIN TENDERMINT NETWORK SUB-SECTION
\subsection{Tendermint Network}\label{tm-network}
\subsubsection{Introduction}\label{tm-network-intro}
\noindent Tendermint cadddn...
\bigskip

\subsubsection{Architecture}\label{tm-network-architecture}
\noindent Architecture can...
\bigskip

\subsubsection{Ethereum peg-zone}\label{tm-network-peg}
\noindent Ethereum can...
\bigskip

\subsubsection{Poster access control}\label{tm-network-access}
\noindent Posters can...2
\bigskip

\subsubsection{State model}\label{tm-network-state}
\noindent State can...
\bigskip

\subsubsection{Transaction types}\label{tm-network-tx-types}
\noindent Transaction can...
\bigskip

\subsubsection{Validator curation}\label{tm-network-validators}
\noindent Validators can...
\bigskip

% END TENDERMINT NETWORK SUB-SECTION
%%%%%%%%%%%%%%%%%%%%%%%%%%%%%%%%%%%%%%%%%%%%%%%%%%%%%%%%%%%%%%%%%%%%%%%%%%%%%%%%

%%%%%%%%%%%%%%%%%%%%%%%%%%%%%%%%%%%%%%%%%%%%%%%%%%%%%%%%%%%%%%%%%%%%%%%%%%%%%%%%
% BEGIN INCENTIVE MODEL SUB-SECTION
\subsection{Incentive Models}\label{incentive-models}
% END INCENTIVE MODEL SUB-SECTION
%%%%%%%%%%%%%%%%%%%%%%%%%%%%%%%%%%%%%%%%%%%%%%%%%%%%%%%%%%%%%%%%%%%%%%%%%%%%%%%%

\clearpage
\pagebreak
% END SPECIFICATION SECTION
%%%%%%%%%%%%%%%%%%%%%%%%%%%%%%%%%%%%%%%%%%%%%%%%%%%%%%%%%%%%%%%%%%%%%%%%%%%%%%%%
%~~~~~~~~~~~~~~~~~~~~~~~~~~~~~~~~~~~~~~~~~~~~~~~~~~~~~~~~~~~~~~~~~~~~~~~~~~~~~~%
%%%%%%%%%%%%%%%%%%%%%%%%%%%%%%%%%%%%%%%%%%%%%%%%%%%%%%%%%%%%%%%%%%%%%%%%%%%%%%%%
\end{document}